%        File: paper1.tex
%     Created: Wed Oct 12 02:00 AM 2011 E
% Last Change: Wed Oct 12 02:00 AM 2011 E
\documentclass[12pt,letterpaper]{article}
\usepackage{setspace}
\usepackage[margin=1.5in]{geometry}
\doublespacing
\begin{document}
\title{Youth Indifference in American Politics: Causes, Consequences,
and Solutions}
\author{Justin Cullen}
\date{\today}
\maketitle

\section{Introduction} % (fold)
\label{sec:Introduction}
It's been a common trend throughout American history for young
Americans to be less involved than their older counterparts in
political issues.  In recent decades however, this gap in the
participation of younger Americans relative to old has noticeably
increased. In polls taken during the 1990s, American's aged 18-29 were
more than 20\% less likely to look forward to reading a newspaper, or
watching news media than those above 30. Data from the 2010 census
shows that American's between 18-29 make up nearly 30\% of the total
American population, so this data suggests that nearly a third of
American citizens show signs of disinterest in political affairs.  If
we accept that civil engagement of citizens is a fundamental necessity
for the functioning of our democratic system, this is a disturbing
trend indeed.  The purpose of this paper is threefold: First, we
establish some of the most important reasons that indifference to
political issues is becoming more widespread among young Americans.
Next, we argue that if left unchecked, this trend could have serious
consequences for Americas' future.  Finally, we offer some steps that
can be taken now to help to reverse this trend and avert these
consequences.
% section Introduction (end)

\section{Why Indifference?} % (fold) 
\label{sec:Why Indifference?}
\subsection{Lack of impact on outcome of political processes}
There are numerous reasons why a young American might refrain from
taking an active interest in following the political events
surrounding him.  The first of these stems from the perceived scale of
the political landscape itself.  Many young Americans feel that their
individual contribution to the outcome of political processes is
negligible, and this can cause them to refrain from participating
in it at all.  Often young people believe that political decisions are
simply made by older politicians, and regardless of what they do, the
outcome will be the same anyways.  This feeling of inevitably is
common particularly among Americans who are just coming into adulthood
and trying to find their place in the world. ``My actions have no
impact, so why do anything?'' is unfortunately a surprisingly feeling
among young voters.

\subsection{Distrust of the Political Process} In addition to being
intimidated by the scale on which political decisions are made, many
young Americans are dissuaded from participating in politics by a
distrust in the integrity of the process itself. In interviews
conducted of a sample of American college students, there was an
overwhelming amount of cynicism regarding the character and
motivations of politicians.  Politicians were often perceived by the
interviewees as ``dishonest'',``corrupt'',``insincere'', or
``self-serving''.  This is an far stronger motivation to refrain from
participating in the political process than the previous example of
size. If someone doesn't have faith that the political outcomes are
determined fairly by the majority will of the people, they have even
less incentive to be engaged in the process than someone who is
deterred by the scale of the system alone.  They might feel that not
only is their individual vote insignificant, but that even if that
manage to gather a sizeable number of supporters, political outcomes
could still be unchanged.

\subsection{Lack of Partisanship} Another factor that contributes to
the young population of America's lack of participation in political
issues is that of partisanship.  Results of polls consistently show a
positive correlation between strong attachment to a political party
and general political involvement.  People with strong partisan views
will naturally have stronger preferences for particular policies and
politicians and thus will be more likely to vote and keep up with
current events. Surveys conducted in the late 1990s found that only
21\% of Americans under 30 considered themselves strong partisans
versus 46\% for Americans above 30. Older Americans are much more
likely to have strong partisan views than young Americans.  This makes
sense - younger Americans simply have had less time to develop rigid
opinions about the world around them and align themselves with
particular causes or parties.  

\subsection{Ineffective Media Reach to Youth}

I would argue that this final reason for youth apathy toward politics
is one of the most important, but it is also the one most easily
remedied.  It's impossible to ignore the huge difference in the amount
that younger Americans follow news media compared to their older
counterparts.  Polls show that while 76\% of Americans over age 30
reported reading a newspaper the day before they were polled, only
31\% of those below age 30 had. Similarly, in the same poll, 86\% of
the over-30 group reported watching a TV news show the day before they
were interviewed, compared with only 45\% of those under 30.  

We must accept the possibility that current primary methods of
communicating current events in politics - namely news TV and paper
newspapers - are simply ineffectively doing the job when it comes to
the younger contingent of voting aged Americans.  One psychological
explanation for this is that children growing up typically associate
reading newspapers and watching the news as something their parents do
- something boring for ``old people''.  Even as children come into
adulthood and reach the age where they themselves should be voting,
this association often remains, causing them to avoid these activities
and thus remained uninformed about current events. Later, I will argue
that the rise of the internet as a medium for communication gives us a
unique opportunity to reach younger audiences without this attached
stigma.  First however, I will discuss the ``why'' of doing this, by
examining some potential consequences of leaving this trend to
continue unchecked.
% section Why Indifference? (end)

\section{Consequences of Youth Indifference} % (fold)
\label{sec:Consequences of Youth Indifference}
On a basic level, the consequences of young Americans' indifference to
public affairs all stem from an assumption that participation in the
political system is a fundamental requirement for being a responsible
citizen in a democratic society.  Brady, Schlozman, Verba convincingly argue the
validity of this assumption in a paper titled ``Civic Participation
and the Equality Problem''.  In this paper, the authors point out that
political inactivity is not uniformly distributed across the
population.  In fact, it is quite the opposite as we have already seen
here in regard to age.  In their paper, they focus on imbalances in political
participation between social classes. They point out the fact that
Americans who are less financially well off are less likely to be
politically involved.  This means that this broad section of the
population is under-represented, and their viewpoints and concerns are
likely to receive less attention and representation than is warranted.
This is in direct opposition to one of the fundamental tenets of
democratic society - namely that the needs and preferences of all
citizens should be equally represented in government.

\subsection{Under-representation of Youth Needs}
I would argue that this analysis applies equally to the case of
age-bias in political involvement.  In the same way as above, a
tendency for political indifference in young voters inadvertently
causes their needs and preferences to be under-represented in
government.  There are many policy decisions which have significantly
more impact on the younger citizens than on those who are middle aged
and above.  Examples of these include healthcare, social security, and
the environment. Over a third of Americans are under age 30 and have
an enormous stake in these types of issues, yet the abstaining from
civic engagement causes their concerns and needs relating to these
issues to be under-emphasized.

\subsection{Impact on Personal Development} A second argument made for
the importance of civic participation in the Brady paper concerns the
development of the individual, rather than the representation of the
groups needs. The Brady paper argues that civic engagement is
paramount in developing many of the desirable characteristics we
associate with responsible citizens.  Among them are independence,
critical thought, respect for others, and a willingness to take
on responsibility. This is of particular concern when applied to the
case of younger Americans. These younger Americans are the future
leaders of the country! If it is indeed true that participation in
ones civic duties are a critical part of learning to become a
respectable adult, then we simply cannot afford to have our youth be
uninvolved.  Of course you might say that it is natural for younger
people to be less interested in politics and that they will ``grow
into'' their roles as responsible citizens with time.  I would argue
that this is not a chance we can afford to take. Amidst a long term
trend of increasing indifference, cannot simply sit back and hope that
young Americans' will ``come around'' in time to lead the country when
there are steps we can take now to combat this trend.


% section Consequences of Youth Indifference (end)



\section{Remedies}
I propose two changes that I feel will help to increase youth
interest in American political affairs.  Of the causes presented in
section 1, I feel that the ineffective use of the media as a
communication medium to young voters is the most tangible, and easiest
to address, so I will focus my attention on that issue.
\end{document}

