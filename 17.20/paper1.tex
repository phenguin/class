%        File: paper1.tex
%     Created: Wed Oct 12 02:00 AM 2011 E
% Last Change: Wed Oct 12 02:00 AM 2011 E
% blah blah
\documentclass[12pt,letterpaper]{article}
\usepackage{setspace}
\usepackage[margin=1.5in]{geometry}
\doublespacing
\begin{document}
\title{Placeholder for a really sweet title I have yet to determine}
\author{Justin Cullen}
\date{\today}
\maketitle
\section{Introduction} % (fold)
\label{sec:Introduction}
It's been a common trend throughout American history for young
Americans to be less involved than their older
counterparts in political issues.  In recent decades however, this gap
in the participation of younger Americans relative to old has
noticeably increased. In polls taken during the 1990s, American's
aged 18-29 were more than 20\% less likely to look forward to reading
a newspaper, or watching news media than those above 30. Data from the
2010 census shows that American's between 18-29 make up nearly 30\% of
the total American population, so this data suggests that nearly a third of American citizens
show signs of disinterest in political affairs.  If we accept that
civil engagement of citizens is a fundamental necessity for the
functioning of our democratic system, this is a disturbing trend
indeed.  The purpose of this paper is threefold: First, we establish
some of the most important reasons that indifference to political
issues is becoming more widespread among young Americans.  Next, we
argue that if left unchecked, this trend could have serious
consequences for Americas' future.  Finally, we offer some steps that
can be taken now to help to reverse this trend and avert these
consequences.
% section Introduction (end)
\section{Why Indifference?} % (fold)
\label{sec:Why Indifference?}

\subsection{Lack of impact on outcome of political processes}
There are numerous reasons why a young American might refrain from
taking an active interest in following the political events
surrounding him.  The first of these stems from the perceived scale of
the political landscape itself.  Many young Americans feel that their
individual contribution to the outcome of political processes is
negligible, and this can cause them to refrain from participating
in it at all.  Often young people believe that political decisions are
simply made by older politicians, and regardless of what they do, the
outcome will be the same anyways.  This feeling of inevitably is
common particularly among Americans who are just coming into adulthood
and trying to find their place in the world. ``My actions have no
impact, so why do anything?'' is unfortunately a surprisingly feeling
among young voters.

\subsection{Distrust of the Political Process} In addition to being
intimidated by the scale on which political decisions are made, many
young Americans are dissuaded from participating in politics by a
distrust in the integrity of the process itself. In interviews
conducted of a sample of American college students, there was an
overwhelming amount of cynicism regarding the character and
motivations of politicians.  Politicians were often perceived by the
interviewees as ``dishonest'',``corrupt'',``insincere'', or
``self-serving''.  This is an far stronger motivation to refrain from
participating in the political process than the previous example of
size. If someone doesn't have faith that the political outcomes are
determined fairly by the majority will of the people, they have even
less incentive to be engaged in the process than someone who is
deterred by the scale of the system alone.  They might feel that not
only is their individual vote insignificant, but that even if that
manage to gather a sizeable number of supporters, political outcomes
could still be unchanged.

\subsection{Lack of Partisanship} Another factor that contributes to
the young population of America's lack of participation in political
issues is that of partisanship.  Results of polls consistently show a
positive correlation between strong attachment to a political party
and general political involvement.  People with strong partisan views
will naturally have stronger preferences for particular policies and
politicians and thus will be more likely to vote and keep up with
current events. Surveys conducted in the late 1990s found that only
21\% of Americans under 30 considered themselves strong partisans
versus 46\% for Americans above 30. Older Americans are much more
likely to have strong partisan views than young Americans.  This makes
sense - younger Americans simply have had less time to develop rigid
opinions about the world around them and align themselves with
particular causes or parties.  

\subsection{}



% section Why Indifference? (end)
\end{document}

